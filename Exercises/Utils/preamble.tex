\documentclass[a4paper, 10pt, final]{scrartcl}


%%%%%%%%%%%%%%%%%%%%%%%%%%%%%%%%%%%%
%%% Create commands for configuration file
%%%%%%%%%%%%%%%%%%%%%%%%%%%%%%%%%%%%
\makeatletter


% \SetLanguage{<language>}. Set the language of the sheet to <language>. Supported languages are english (default) and ngerman. The current language is stored in the \ks@lan macro.
\newcommand{\SetLanguage}[1]{%
\providecommand\ks@lan{#1}%
}


% Create the switch for checking whether the title has changed (thc). If there is no custom title, the switch will have the value FASLE. This detection is needed to make a translateable default title which can be customized as well.
\newif\if@ks@thc@\@ks@thc@false


% \SetTitleName{<name>}. Set the title name of the sheet to <name>. It only changes the name of the title, the number will add anyway and can't be removed. The title name will strored in \ks@title@name and the \if@ks@thc@ switch will globaly set to TRUE.
\newcommand{\SetTitleName}[1]{%
\providecommand\ks@title@name{#1}%
\@ks@thc@true%
}


% \SetSubTitle{<sub>}. Set the subtitle of the sheet to <sub>. An empty subtitle is the default value. You can use \GetDeadline to get the day and time of delivery.
\newcommand{\SetSubTitle}[1]{%
\providecommand\ks@subtitle{#1}%
}

\newcommand{\SetFooter}[1]{%
\providecommand\ks@footer{#1}%
}


% \SetDateOfFirstUpload{<year>}{<month>}{<day>}. Set the day of the first upload to <year>-<month>-<day>. The format of the date depends on the language of the document. You should set the date for consitency. Otherwise the automatic date calculation might fail.
\newcommand{\SetDateOfFirstUpload}[3]{%
\SaveDate%
\day=#3%
\month=#2%
\year=#1%
}


% \SetUploadFrequence{<days>}. Set the frequency of upload in days. The default value is 7, i.e. one upload every week.
\newcommand{\SetUploadFrequence}[1]{%
\providecommand\ks@upf{#1}%
}


% \SetDaysUntilDeadline{<days>}. Set the number of days until which the homework has to be delivered. The default value is 7, i.e. the deadline will be one week after the upload of the sheet.
\newcommand{\SetDaysUntilDeadline}[1]{%
\providecommand\ks@dud{#1}%
}


% \SetDeadlineTime{<time>}. Set the time of the deadline. The default value is 23:59.
\newcommand{\SetDeadlineTime}[1]{%
\providecommand\ks@dlt{#1}%
}


% \SetLecturer{<lecturer>}. Set the lecturer of the lecture and the the tutorial. The value can be any string, for example, \SetLecturer{Prof. Dr. J. Doe, Dr. J. Roe}
\newcommand{\SetLecturer}[1]{%
\providecommand\ks@lecturer{#1}%
}


% \SetLecture{<lecture>}. Set the lecture for which the exercises will be for. The value can be any string, for example, \SetLecture{Analysis I}
\newcommand{\SetLecture}[1]{%
\providecommand\ks@lecture{#1}%
}


% \SetTerm{<term>}. Set the value of the current term. The value can be any string, for example, \SetTerm{WT 22/23}
\newcommand{\SetTerm}[1]{%
\providecommand\ks@term{#1}%
}


% \SetOffsetAfterSheet{<offset>}{<sheet>}. Set an offset of <offset> days after sheet <sheet>. This can be usefull if there are holidays in between the current term. For example, if there are two weeks of holidays before sheet 10 will be uploaded, you should use \SetOffsetAfterSheet{14}{9}.
\newcommand{\SetOffsetAfterSheet}[2]{%
\providecommand\ks@offset{#1}%
\providecommand\ks@offset@sheet{#2}%
}


%%%%%%%%%%%%%%%%%%%%%%%%%%%%%%%%%%%%
%%% Usepackages
%%%%%%%%%%%%%%%%%%%%%%%%%%%%%%%%%%%%


% Default math stuff. Loads amsmath. See https://www.ctan.org/pkg/mathtools
\usepackage{mathtools}


% Add additional math symbols.
\usepackage{amssymb}%TODO For what exactly is it needed


% Add font to define the indicator function. See https://www.ctan.org/tex-archive/fonts/doublestroke
\usepackage{dsfont}


% For color, images and graphics. See https://www.ctan.org/pkg/latex-graphics
\usepackage{color}
\usepackage{graphics}


% Add german language support. Allow special caracters such as ä,ö, etc. See https://www.typeerror.org/docs/latex/fontenc-package and https://www.ctan.org/pkg/inputenc
\usepackage[T1]{fontenc}
\usepackage[utf8]{inputenc}

% Make the look of the text prettier. See https://www.namsu.de/Extra/pakete/Lmodern.html
\usepackage{lmodern}


% For better referencing. Espacially for translateable page ref. See https://ctan.org/pkg/cleveref
\usepackage{cleveref}


% For the definition of the exercise and solution enviroment. See https://www.ctan.org/pkg/environ and https://www.ctan.org/pkg/ntheorem
\usepackage{environ}
\usepackage[amsmath,amsthm]{ntheorem}


% For specification of the page layout. Header size, footsize, etc. See https://www.ctan.org/pkg/geometry
\usepackage[bottom=3.7cm, top=2cm]{geometry}


% Increase space between two lines. See https://www.ctan.org/pkg/setspace
\usepackage[onehalfspacing]{setspace}


% For an enviroment with multiple columns. See https://www.ctan.org/pkg/multicol
\usepackage{multicol}


% For date calculation. See https://www.ctan.org/pkg/scrdate and https://www.ctan.org/pkg/advdate
\usepackage{scrdate}
\usepackage{advdate}


% For the definition of the header and footer.
\usepackage[headsepline,footsepline,singlespacing]{scrlayer-scrpage}

%%%%%%%%%%%%%%%%%%%%%%%%%%%%%%%%%%%%
%%% Input configuration file
%%%%%%%%%%%%%%%%%%%%%%%%%%%%%%%%%%%%
\makeatother
\SetLanguage{ngerman}

\SetLecturer{}
\SetLecture{}
\SetTerm{}

%\SetTitleName{Exercise Sheet}
%\SetSubTitle{Please deliver your solution of the homework until \GetDeadline.}
%SetFooter{Please deliver your solution of the homework until \GetDeadline.}

\SetDateOfFirstUpload{2022}{10}{25}
\SetUploadFrequence{7}
\SetDaysUntilDeadline{7}
\SetDeadlineTime{11:00}

%\SetOffsetAfterSheet{<days>}{<sheet>}



%%%%%%%%%%%%%%%%%%%%%%%%%%%%%%%%%%%%
%%% Set the default configuration values
%%%%%%%%%%%%%%%%%%%%%%%%%%%%%%%%%%%%
\makeatletter


\SetLanguage{english}
% Load the babel package immideately after the language is set. And befor the title is set. See https://www.ctan.org/pkg/babel
\usepackage[\ks@lan]{babel}
\if@ks@thc@\else
\SetTitleName{Exercise Sheet}%
\defcaptionname{ngerman}{\ks@title@name}{Übungsblatt}%
\fi


\SetSubTitle{}
\SetFooter{}
\SetUploadFrequence{7}
\SetDaysUntilDeadline{7}
\SetDeadlineTime{23:59}
\SetLecturer{}
\SetLecture{}
\SetTerm{}
\SetOffsetAfterSheet{0}{0}


%%%%%%%%%%%%%%%%%%%%%%%%%%%%%%%%%%%%
%%% General settings
%%%%%%%%%%%%%%%%%%%%%%%%%%%%%%%%%%%%


%
\addtokomafont{pageheadfoot}{\upshape}


%
\pagestyle{scrheadings}


%
\setlength{\headheight}{5.5em}
\setlength{\footheight}{2em}
\setlength{\footskip}{10.5mm}

\ModifyLayer[addvoffset=-.3em]{scrheadings.foot.above.line}


% Center a colon to the mathematical axis. An ordinary colon can be output with \ordinarycolon
\mathtoolsset{centercolon}
\newcommand{\coloneqq}{:=}
\renewcommand{\colon}{:}


% Set the indent for new paragraphs to 0, so there will be no indent.
\setlength{\parindent}{0em}


% Definie the standard path where images will be searched.
\graphicspath{{./Images}}


% Declare the default graphic extensions. If there are files with the same name but different file extensions, the .pdf file will be preffered.
\DeclareGraphicsExtensions{.pdf,.png,.jpg}


%%%%%%%%%%%%%%%%%%%%%%%%%%%%%%%%%%%%
%%% Get Date commands
%%%%%%%%%%%%%%%%%%%%%%%%%%%%%%%%%%%%


% Declare date of uploading and date of deadline. At this moment the day of first upload is stroed in the macros
\SaveDate[\ks@upload]
\SaveDate[\ks@deadline]


% \GetUpload contains the day of upload for usage in the document. Note that the macro \today will be overwrited by the date of upload. The result will be printed in the format "<nameOfDay>, <dateOfUpload>".
\newcommand{\GetUpload}{%
\SetDate[\ks@upload]%
\DayName{\year}{\month}{\day}, \today%
}


% \GetDeadline contains the day and time of the deadline for usage in the document. Note that the macro \today will be overwrited by the date of deadline. The result will be printed in the format "<nameOfDay>, <dateOfDeadline>, <timeOfDeadline>".
\newcommand{\GetDeadline}{%
\SetDate[\ks@deadline]%
\DayName{\year}{\month}{\day}, \today, \ks@dlt%
}


%%%%%%%%%%%%%%%%%%%%%%%%%%%%%%%%%%%%
%%% Counter
%%%%%%%%%%%%%%%%%%%%%%%%%%%%%%%%%%%%


% Declare a counter for the sheets. The value of the counter has to be set in every sheet seperately with the \SetNumber command.
\newcounter{num}


% \SetNumber{<value>}. Set the number of the sheet to <value>. You have to set the number on each sheet. After the number is set, the upload date and the deadline will be calculated
\newcommand{\SetNumber}[1]{%
\setcounter{num}{#1}%set counter
% caculate dates
\def\ks@upload@offset{0}% declare upload offset
\def\ks@deadline@offset{0}% declare deadline offset
\ifnum\ks@offset@sheet<\thenum% check whether to apply the offset
\def\ks@upload@offset{\ks@offset}% overwrite the offset if neccessary
\fi%
\AdvanceDate[\the\numexpr(\thenum-1)*\ks@upf+\ks@upload@offset\relax]% increase the date of first upload by the number of days needed for the current sheet
\SaveDate[\ks@upload]% save the current day, which is "DateOfFirstUpload + x" at the \ks@upload macro
\ifnum\ks@offset@sheet=\thenum% check whether to apply an offset to the deadline
\def\ks@deadline@offset{\ks@offset}% overwrite the offset if neccessary
\fi%
\AdvanceDate[\the\numexpr\ks@dud+\ks@deadline@offset\relax]% increase the date of current upload by the days until delivery
\SaveDate[\ks@deadline]% save the current day, which is "DateOfFirstUpload + x + dud" at the \ks@deadline macro
}


%%%%%%%%%%%%%%%%%%%%%%%%%%%%%%%%%%%%
%%% Header and Footer
%%%%%%%%%%%%%%%%%%%%%%%%%%%%%%%%%%%%


% Right side of the header
\ohead{%
\ks@lecture\ -- \ks@term \\%
\ks@lecturer\ \\%
\GetUpload%
}


% Left side of the header
\ihead{\includegraphics{logo}}


% Left side of the footer
\ifoot{\smash{\vtop{\ks@footer}}}


% Remove the page numbering in the center of the footer
\cfoot{}


% Right side of the footer. NOTE: Here the custom command \tof is used, which is the translateable word of "of".
\ofoot{%
\Cref@page@name\ \thepage \tof\ \labelcpageref{lastpage}%
}


% Print title
\AtBeginDocument{%
\begin{center}%
 \LARGE \textbf{\ks@title@name\ \thenum} \\%
 \normalsize \ks@subtitle%
\end{center}%
}

\newcommand{\LastNote}{%
\begin{center}%
\feedback%
\\\textbf{\gl}%
\end{center}%
}


% Create Reference of the lat page. TODO: Known bug, if the page is to full, there might be not the right reference of the last page.
\AtEndDocument{%
\vfill%
\LastNote
\label{lastpage}%
}


%%%%%%%%%%%%%%%%%%%%%%%%%%%%%%%%%%%%
%%% Translateable text
%%%%%%%%%%%%%%%%%%%%%%%%%%%%%%%%%%%%


% \DeclareTMT{<name>}{<english>}{<german>} Create a macro \text<name> which is translatable. Additionally create a macro \t<name> which is the translateable text in a text enviroment and a space infront. To make a space after the text use \t<name>\ or \t<name>{}
\newcommand{\DeclareTMT}[3]{%
\edef\tmp{{ngerman}{\csname text#1\endcsname}{#3}}
\expandafter\def\csname text#1\endcsname{#2}%
\expandafter\defcaptionname\tmp%
\expandafter\def\csname t#1\endcsname{\text{ \csname text#1\endcsname}}%
}


\DeclareTMT{on}{on}{auf}
\DeclareTMT{of}{of}{von}
\DeclareTMT{and}{and}{und}
\DeclareTMT{or}{or}{oder}
\DeclareTMT{in}{in}{in}
\DeclareTMT{for}{for}{für}
\DeclareTMT{forall}{for all}{für alle}
\DeclareTMT{with}{with}{mit}
\DeclareTMT{if}{if}{falls}
\DeclareTMT{else}{else}{sonst}
\DeclareTMT{even}{even}{gerade}
\DeclareTMT{odd}{odd}{ungerade}


% Make "Page" translatable
\defcaptionname{ngerman}{\Cref@page@name}{Seite}

% Make "Points" translatable
\def\points{points}
\defcaptionname{ngerman}{\points}{Punkte}

% Make "Good luck" translatable
\def\gl{Good luck!}
\defcaptionname{ngerman}{\gl}{Viel Erfolg!}

% Make the Feedback translatable
\def\feedback{Please inform us about any typos you found.}
\defcaptionname{ngerman}{\feedback}{Teilen Sie uns gefundene Tippfehler bitte gerne mit.}



%%%%%%%%%%%%%%%%%%%%%%%%%%%%%%%%%%%%
%%% Solution switches and commands
%%%%%%%%%%%%%%%%%%%%%%%%%%%%%%%%%%%%


% \ks@sol@save{<arg>}. Contain <arg> if and only if the current solution key \ks@solkey is TRUE. Whenever you want to execute <arg> only in the case when solutions are enabled use this command.
\newcommand{\ks@sol@save}[1]{%
\expandafter\csname if@ks@\ks@solkey\endcsname%
#1%
\fi%
}


% \IfSolution[<key>]{<code>} Execute <code> if and only if the solutions of the given key <key> (default: exercise) is enabled.
\newcommand{\IfSolution}[2][exercise]{%
\def\ks@solkey{#1}% set the given key
\ks@sol@save{#2}% execute code save
\def\ks@solkey{}% clear the given key
}


% \NewpageIfSolution[<key>] Make a new page if and only if the solution for the key <key> (default: exercise) is enabled.
\newcommand{\NewpageIfSolution}[1][exercise]{%
\IfSolution[#1]{\newpage}%
}


% \IfSolutionDisabled[<key>]{<code>} Execute <code> if and only if the solution for the key <key> (default: exercise) is disabled.
\newcommand{\IfSolutionDisabled}[2][exercise]{%
\expandafter\csname if@ks@#1\endcsname\else%
#2%
\fi%
}


% \NewpageIfSolutionDisabled[<key>] Make a new page if and only if the solution for the key <key> (default: exercise) is disabled.
\newcommand{\NewpageIfSolutionDisabled}[1]{%
\IfSolutionDisabled[#1]{\newpage}%
}


%%%%%%%%%%%%%%%%%%%%%%%%%%%%%%%%%%%%
%%% Enviroments styles
%%%%%%%%%%%%%%%%%%%%%%%%%%%%%%%%%%%%


% Set the font of the body of a theorem to normal font.
\theorembodyfont{}


% Set the separator between theorem header and the optional argument to ":".
\theoremseparator{:}


% Set the spacing before a theorem
\theorempreskip{\topsep}


% Set the spacing after a theorem
\theorempostskip{2\topsep}


% \ks@headerstyle[<opt>]{<header>} Define custome header style so that the exercises automatically start in the new line. The Style will be like: "<header>:<opt>"
\newcommand\ks@headerstyle[2][]{%
\item[\rlap{\vbox{\hbox{\hskip\labelsep\theorem@headerfont
#2\theorem@separator #1}\hbox{\strut}}}]%
}


% Define the exercise theoremstyle
\newtheoremstyle{exercise}{%
\ks@headerstyle{##1\ ##2}% without optional argument
}{%
\ks@headerstyle[{~ ##3}]{##1\ ##2}% with optional argument
}


% Define the nonumber theoremstyle. E.g. for solution or extra tasks.
\newtheoremstyle{nonumber}{%
\ks@headerstyle{##1}% without optional argument
}{%
\ks@headerstyle[~ ##3]{##1}% with optional argument
}


% Define the hint theoremstyle where the body doens't start in a new line. There is no optinal argument allowed.
\newtheoremstyle{hint}{%
\item[\theorem@headerfont\hskip\labelsep ##1\theorem@separator]%
}{%
\item[\theorem@headerfont\hskip \labelsep ##1\theorem@separator]%
}


%%%%%%%%%%%%%%%%%%%%%%%%%%%%%%%%%%%%
%%% Translatable enviroments
%%%%%%%%%%%%%%%%%%%%%%%%%%%%%%%%%%%%


% \NewTranslatableEnviroment{<key>}{<english>}{<englishpl>}{<german>}{<germanpl>}
\newcommand{\NewTranslatableEnviroment}[5]{%
%
\edef\tmpsin{{ngerman}{\csname #1autorefname\endcsname}{#4}}
\edef\tmpplu{{ngerman}{\csname #1autorefnames\endcsname}{#5}}
\edef\tmp{{#1}{\csname #1autorefname\endcsname}{\csname #1autorefnames\endcsname}}


\expandafter\def\csname #1autorefname\endcsname{#2}% Define the name of the translatable enviroment
%
\expandafter\def\csname #1autorefnames\endcsname{#2}% Define the plural name of the enviroment (only for use with cleverref)
%
\expandafter\newtheorem{#1}{\csname #1autorefname\endcsname}[num]% Define the enviroment
%
\expandafter\defcaptionname\tmpsin% Make the caption translatable to german
%
\expandafter\defcaptionname\tmpplu% Make the plural translatable (only for use with cleverref)
%
\expandafter\crefname\tmp% Tell cref the caption and its plural.
}


%%%%%%%%%%%%%%%%%%%%%%%%%%%%%%%%%%%%
%%% Exercise like enviroments
%%%%%%%%%%%%%%%%%%%%%%%%%%%%%%%%%%%%


% \Enable{<keys>} Enable the solution of an exerciselike enviroment with keys <keys> defined by \NewExerciseLikeEnviroment. Teh argument <keys> has to be a list seperatetd by ",". In addition to the exerciselike enviroments you can enable "note" for a lecturer version.
\newcommand{\Enable}[1]{%
\ks@enable #1,\relax\noexpand\@eolst%
}

\def\ks@enable #1,#2\@eolst{%
\ifx\relax#2\relax%
\expandafter\global\csname @ks@#1true\endcsname%
\else%
\expandafter\global\csname @ks@#1true\endcsname%
\ks@enable #2\@eolst%
\fi%
}


% \NewExerciseLikeEnviroment[<style>]{<key>}{<english>}{<englishpl>}{<german>}{<germanpl>} Create a new enviroment <key> with style <style> (default: exercise). The displayed name of the enviroment is translatable. You are able to use \begin{<key>}, \end{<key>} and \begin{solution}\end{solution} within it. The solution only will displayed for all enviroments of the type <key> when you call \EnableSolution{<key>} before.
\newcommand{\NewExerciseLikeEnviroment}[6][exercise]{%
%
\expandafter\newif\csname if@ks@#2\endcsname% Define the switch
%
\expandafter\csname @ks@#2false\endcsname% Set the default value to false
%
\theoremstyle{#1}% Set the theoremstyle
%
\theoremprework{\def\ks@solkey{#2}}% Set the solution key at the beginning of the enviroment
%
\theorempostwork{\def\ks@solkey{}}% Reset the solution key at the end of the enviroment
%
\NewTranslatableEnviroment{#2}{#3}{#4}{#5}{#6}
}


\NewExerciseLikeEnviroment{task}{Presence task}{Presence tasks}{Präsenzaufgabe}{Präsenzaufgaben}

\NewExerciseLikeEnviroment{homework}{Homework}{Homework}{Hausaufgabe}{Hausaufgaben}

\NewExerciseLikeEnviroment{exercise}{Exercise}{Exercises}{Aufgabe}{Aufgaben}

\NewExerciseLikeEnviroment[nonumber]{extra}{Extra}{Extra}{Zusatzaufgabe}{Zusatzaufgaben}

% \NewExerciseLikeEnviroment[nonumber]{collection}{Col}{Col}{Col}{Col}


%%%%%%%%%%%%%%%%%%%%%%%%%%%%%%%%%%%%
%%% Solution like enviroments
%%%%%%%%%%%%%%%%%%%%%%%%%%%%%%%%%%%%


% Set the enviroment style
\theoremstyle{nonumber}


% Declare the Solution enviroment
\NewTranslatableEnviroment{solutionThm}{Solution sketch}{Solution sketches}{Lösungshinweis}{Lösungshinweise}


% Make the solution enviroment save for the solution keys, i.e. only print the solution if it is enabled.
\NewEnviron{solution}{%
\ks@sol@save{\begin{solutionThm}\BODY}%
}[%
\ks@sol@save{\end{solutionThm}}%
]


% Define Solution hints for useing inline.
\newcommand{\inlineSolution}[1]{%
\ks@sol@save{(#1)}%
}


%%%%%%%%%%%%%%%%%%%%%%%%%%%%%%%%%%%%
%%% Hint like enviroments
%%%%%%%%%%%%%%%%%%%%%%%%%%%%%%%%%%%%


% Set the enviroment style
\theorembodyfont{\itshape}
\theorempreskip{0pt}
\theorempostskip{0pt}
\theoremstyle{hint}


\NewTranslatableEnviroment{hint}{Hint}{Hints}{Hinweis}{Hinweise}

\NewTranslatableEnviroment{info}{Information}{Information}{Information}{Informationen}


%%%%%%%%%%%%%%%%%%%%%%%%%%%%%%%%%%%%
%%% Itemize like enviroments
%%%%%%%%%%%%%%%%%%%%%%%%%%%%%%%%%%%%


% Redefine the enumeration for subexercises
\renewcommand{\labelenumi}{\alph{enumi})}
\renewcommand{\labelenumii}{\roman{enumii})}


% Item label for wrong (w) answers in quizes
\def\ks@wlable{$\square$}


% Item label for right (r) answers. By default it looks the same as the wrong item. But if the solution is enabled it will change its shape to a checked square.
\def\ks@rlable{\ks@wlable}


% \itemr Define a new item to use for right (r) answers in quizes. The item label will change to a checked square if the solution is enabled.
\newcommand{\itemr}{\item[\ks@rlable]}


% \itemw Define a new item to use for wrong (w) answers in quizes. It is the same as \item in quiz enviroments. It is just for consitency.
\newcommand{\itemw}{\item[\ks@wlable]}


% Define the quiz enviroment
\newenvironment{quiz}{%
%
\begin{itemize}% start a normal itemize enviroment
%
\renewcommand{\labelitemi}{\ks@wlable}% redefine the standard item label
%
\ks@sol@save{\def\ks@rlable{$\text{\rlap{$\checkmark$}}\square$}}% redefine the label for the right answers (\itemr) if and only if the solution is enabled
}{%
\end{itemize}% end the itemize enviroment
}


%%%%%%%%%%%%%%%%%%%%%%%%%%%%%%%%%%%%
%%% Notes for tutors
%%%%%%%%%%%%%%%%%%%%%%%%%%%%%%%%%%%%

% Define the switch and set it to false
\newif\if@ks@note\@ks@notefalse

%
\newcommand{\note}[2][]{%
\if@ks@note{%
\renewcommand\thefootnote{\textcolor{red}{\arabic{footnote}}}%
\ifx\relax#2%
\else\textcolor{red}{#2}%
\fi%
\ifx\relax#1\else%
\footnote{\textcolor{red}{#1}}%
\fi}%
\fi%
}


\makeatother

%%%%%%%%%%%%%%%%%%%%%%%%%%%%%%%%%%%%
%%% Symbols
%%%%%%%%%%%%%%%%%%%%%%%%%%%%%%%%%%%%


%%% Greek letters
\renewcommand{\epsilon}{\varepsilon}
\renewcommand{\phi}{\varphi}
\renewcommand{\theta}{\vartheta}
\renewcommand{\rho}{\varrho}


%%% Upper d for integrals
\renewcommand{\d}{\mathop{}\!\mathrm{d}}


%%% Infinity
\newcommand{\oo}{\infty}


%%% sigma-
\newcommand{\sigmad}{\ensuremath{\sigma}-}


%%% ^star
\renewcommand{\star}{\ensuremath{^\ast}}


%%% Dot to use in functions. 
\newcommand{\fdot}{\:\cdot\:}


%%% Number spaces
\newcommand{\field}[1]{\mathbb{#1}}
\newcommand{\numN}{\field N}
\newcommand{\N}{\numN}
\newcommand{\numNo}{\numN_0}

\newcommand{\numZ}{\field Z}
\newcommand{\Z}{\numZ}

\newcommand{\numQ}{\field Q}
\newcommand{\Q}{\numQ}

\newcommand{\numR}{\field R}
\newcommand{\R}{\numR}
\newcommand{\numRp}{\numR^+}
\newcommand{\numRpo}{\numRp_0}

\newcommand{\numC}{\field C}
\newcommand{\C}{\numC}

\newcommand{\numK}{\field K}
\newcommand{\K}{\numK}


%%% Sets
\renewcommand{\complement}{\mathsf{c}}
\newcommand{\comp}{\complement}
\newcommand{\closure}[1]{\overline{#1}}
\newcommand{\inner}[1]{#1^\circ}
\newcommand{\restrict}[1]{\vert_{#1}}
\newcommand{\subsubset}{\subset\subset}
\newcommand{\relcomp}{\subsubset}
\newcommand{\embed}{\hookrightarrow}
\newcommand{\cembed}{\embed\embed}
\newcommand{\cupd}{\mathbin{\mathaccent\cdot\cup}}
\newcommand{\cupdot}{\cupd}


%%% Symbols with limits
% Convergence
\newcommand{\conv}[1][]{\xrightarrow{\:#1\:}}
\newcommand{\convn}[1][n]{\conv[#1\to\oo]}
\newcommand{\wconv}[1][]{\xrightharpoonup{\:#1\:}}
\newcommand{\wconvn}[1][n]{\wconv[#1\to\oo]}

% Lim
\newcommand{\limn}[1][n]{\lim_{#1\to\infty}}
\newcommand{\limnl}[1][n]{\lim\limits_{#1\to\infty}}

% Liminf
\newcommand{\liminfn}[1][n]{\liminf_{#1\to\infty}}
\newcommand{\liminfnl}[1][n]{\liminf\limits_{#1\to\infty}}

% Limsup
\newcommand{\limsupn}[1][n]{\limsup_{#1\to\infty}}
\newcommand{\limsupnl}[1][n]{\limsup\limits_{#1\to\infty}}

% Inf
\newcommand{\infn}[1][n]{\inf_{#1\in\numN}}
\newcommand{\infnl}[1][n]{\inf\limits_{#1\in\numN}}

% Sup
\newcommand{\supn}[1][n]{\sup_{#1\in\numN}}
\newcommand{\supnl}[1][n]{\sup\limits_{#1\in\numN}}

% Sum
\newcommand{\sumn}[1][n]{\sum_{#1=1}^\oo}
\newcommand{\sumnl}[1][n]{\sum\limits_{#1=1}^\oo}


%%% Function spaces
% Continuous functions
\newcommand{\funC}{\mathcal C}
\newcommand{\funCk}[1][k]{\funC^{#1}}
\newcommand{\funCko}[1][k]{\funCk[#1]_0}
\newcommand{\funCo}{\funCk[0]}
\newcommand{\funCoo}{\funCk[\oo]}
\newcommand{\funCooo}{\funCko[\oo]}
\newcommand{\funClip}{\funCk[\text{Lip}]}
\newcommand{\funCb}{\funC_\text{b}}
\newcommand{\funCkg}[1][k,\gamma]{\funCk[#1]}
\newcommand{\funCog}[1][0,\gamma]{\funCk[#1]}

% Polynomials
\newcommand{\funP}{\mathcal P}
\newcommand{\funPn}[1][n]{\funP_{#1}}
    
% Lp
\newcommand{\funL}{L}
\newcommand{\funLp}[1][p]{\funL^{#1}}
\newcommand{\funLoo}{\funLp[\oo]}
\newcommand{\funLploc}[1][p]{\funL^{#1}_\text{loc}}

\newcommand{\funl}{\mathcal L}
\newcommand{\funlp}[1][p]{\funl^{#1}}
\newcommand{\funloo}{\funlp[\oo]}
\newcommand{\funlploc}[1][p]{\funl^{#1}_\text{loc}}
    
% Wkp
\newcommand{\funW}{W}
\newcommand{\funWkp}[1][k,p]{\funW^{#1}}
\newcommand{\funWkpo}[1][k,p]{\funWkp[#1]_0}
\newcommand{\funWkploc}[1][k,p]{\funWkp[#1]_\text{loc}}
    
% Hilbert spaces
\newcommand{\funH}{H}
\newcommand{\funHk}[1][k]{\funH^{#1}}
\newcommand{\funHko}[1][k]{\funH[#1]_0}

% Distributions
\newcommand{\funD}{\mathcal D}
\newcommand{\funS}{\mathcal S}

% Factor space
\newcommand{\factor}[2]{{}^{#1}\!/\!_{#2}}

% Sequence spaces
\newcommand{\seqlp}[1][p]{\ell^{#1}}
\newcommand{\seqwkp}[1][k,p]{w^{#1}}
\newcommand{\seqc}{c}
\newcommand{\seqco}{\seqc_0}
\newcommand{\seqd}{d}


%%%Integral

% \Xint{symbol} integral with symbol in it
\def\Xint#1{\mathchoice
{\XXint\displaystyle\textstyle{#1}}%
{\XXint\textstyle\scriptstyle{#1}}%
{\XXint\scriptstyle\scriptscriptstyle{#1}}%
{\XXint\scriptscriptstyle\scriptscriptstyle{#1}}%
\!\int}
\def\XXint#1#2#3{{\setbox0=\hbox{$#1{#2#3}{\int}$} \vcenter{\hbox{$#2#3$}}\kern-.5\wd0}}

% Averrage integral
\def\aint{\Xint{\text{\hspace{.005\wd0}\tiny\rotatebox[origin=c]{-45}{/}}}}


%%%%%%%%%%%%%%%%%%%%%%%%%%%%%%%%%%%%
%%% Math operators
%%%%%%%%%%%%%%%%%%%%%%%%%%%%%%%%%%%%


% Real part of a complex number
\let\Re\relax
\DeclareMathOperator{\Re}{Re}

% Imaginary part of a complex number
\let\Im\relax
\DeclareMathOperator{\Im}{Im}

% Signum function
\DeclareMathOperator{\sgn}{sgn}

% Linear Hull
\DeclareMathOperator{\lin}{span}

% Arcus cotangens
\DeclareMathOperator{\arccot}{arccot}

% Range
\DeclareTMT{ran}{ran}{Bild}% Defines \textran
\DeclareMathOperator{\ran}{\textran}

% Support
\DeclareMathOperator{\supp}{supp}

% Trace
\DeclareTMT{tr}{tr}{spur}
\DeclareMathOperator{\tr}{\texttr}

% Identity
\DeclareMathOperator{\id}{id}

% Rank
\DeclareTMT{rk}{rk}{rg}
\DeclareMathOperator{\rk}{\textrk}

% Distance
\DeclareMathOperator{\dist}{dist}

% Ddiameter
\DeclareMathOperator{\diam}{diam}

% Indicator function
\DeclareMathOperator{\ind}{\mathds{1}}

% Cofactor matrix
\DeclareMathOperator{\cof}{cof}

% Accumulation values
\DeclareTMT{av}{av}{HW}
\DeclareMathOperator{\av}{\textav}

% Accumulation points
\DeclareTMT{ap}{ap}{HP}
\DeclareMathOperator{\ap}{\textap}

% Gradient
\DeclareMathOperator{\grad}{\nabla}
\DeclareMathOperator{\tgrad}{grad}

% Divergence
\let\div\relax
\DeclareMathOperator{\div}{\grad \cdot}
\DeclareMathOperator{\tdiv}{div}

% Curl product
\newcommand{\curl}{\times}

% Rotation
\DeclareMathOperator{\rot}{\grad \curl}
\DeclareMathOperator{\trot}{rot}

% Laplace
\DeclareMathOperator{\laplace}{\Delta}


%%%%%%%%%%%%%%%%%%%%%%%%%%%%%%%%%%%%
%%% Paired delimiters
%%%%%%%%%%%%%%%%%%%%%%%%%%%%%%%%%%%%
\makeatletter


% Absolut value 
\DeclarePairedDelimiterXPP{\absDel}[2]{}{\lvert}{\rvert}{_{#2}}{#1}
\def\abs{\@ifstar\@abs\@@abs}
\def\absD{\@ifstar\@absD\@@absD}
\newcommand{\@@abs}[2][]{\absDel*{#2}{#1}}
\newcommand{\@@absD}[1][]{\absDel*{\fdot}{#1}}
\newcommand{\@abs}[2][]{\absDel{#2}{#1}}
\newcommand{\@absD}[1][]{\absDel{\fdot}{#1}}

% Norm with ||
\DeclarePairedDelimiterXPP{\normDel}[2]{}{\lVert}{\rVert}{_{#2}}{#1}
\def\norm{\@ifstar\@norm\@@norm}
\def\normD{\@ifstar\@normD\@@normD}
\def\normop{\@ifstar\@normop\@@normop}
\def\normopD{\@ifstar\@normopD\@@normopD}
\newcommand{\@@norm}[2][]{\normDel*{#2}{#1}}
\newcommand{\@@normD}[1][]{\normDel*{\fdot}{#1}}
\newcommand{\@@normop}[1][]{\normDel*{#1}{\text{op}}}
\newcommand{\@@normopD}{\normDel*{\fdot}{\text{op}}}
\newcommand{\@norm}[2][]{\normDel{#2}{#1}}
\newcommand{\@normD}[1][]{\normDel{\fdot}{#1}}
\newcommand{\@normop}[1]{\normDel{#1}{\text{op}}}
\newcommand{\@normopD}{\normDel{\fdot}{\text{op}}}

% Norm with |||
\DeclareFontFamily{U}{matha}{\hyphenchar\font45}
\DeclareFontShape{U}{matha}{m}{n}{
 <5> <6> <7> <8> <9> <10> gen * matha <10.95> matha10 <12> <14.4> <17.28> <20.74> <24.88> matha12
}{}
\DeclareSymbolFont{matha}{U}{matha}{m}{n}
\DeclareFontSubstitution{U}{matha}{m}{n}
\DeclareFontFamily{U}{mathx}{\hyphenchar\font45}
\DeclareFontShape{U}{mathx}{m}{n}{
 <5> <6> <7> <8> <9> <10> <10.95> <12> <14.4> <17.28> <20.74> <24.88> mathx10
}{}
\DeclareSymbolFont{mathx}{U}{mathx}{m}{n}
\DeclareFontSubstitution{U}{mathx}{m}{n}
\DeclareMathDelimiter{\vvvert}{0}{matha}{"7E}{mathx}{"17}
\DeclarePairedDelimiterXPP{\NormDel}[2]{}{\vvvert}{\vvvert}{_{#2}}{#1}
\def\Norm{\@ifstar\@Norm\@@Norm}
\def\NormD{\@ifstar\@NormD\@@NormD}
\newcommand{\@@Norm}[2][]{\NormDel*{#2}{#1}}
\newcommand{\@@NormD}[1][]{\NormDel*{\fdot}{#1}}
\newcommand{\@Norm}[2][]{\NormDel{#2}{#1}}
\newcommand{\@NormD}[1][]{\NormDel{\fdot}{#1}}

% Scalar product
\DeclarePairedDelimiterXPP{\innerprodDel}[3]{}{\langle}{\rangle}{_{#3}}{#1,#2}
\def\innerprod{\@ifstar\@innerprod\@@innerprod}
\def\innerprodD{\@ifstar\@innerprodD\@@innerprodD}
\newcommand{\@@innerprod}[3][]{\innerprodDel*{#2}{#3}{#1}}
\newcommand{\@@innerprodD}[1][]{\innerprodDel*{\fdot}{\fdot}{#1}}
\newcommand{\@innerprod}[3][]{\innerprodDel{#2}{#3}{#1}}
\newcommand{\@innerprodD}[1][]{\innerprodDel{\fdot}{\fdot}{#1}}


% Commutator
\DeclarePairedDelimiterX{\commuDel}[2]{\lbrack}{\rbrack}{#1,#2}
\newcommand{\commutator}[2]{\commuDel*{#1}{#2}}
\newcommand{\commutatorbf}[2]{\commuDel*{\mathbf #1}{\mathbf #2}}


% Bra-ket
\DeclarePairedDelimiter{\braDel}{\langle}{\vert}
\newcommand{\bra}[1]{\braDel*{#1}}
\newcommand{\braD}{\bra{\fdot}}
\DeclarePairedDelimiter{\ketDel}{\vert}{\rangle}
\newcommand{\ket}[1]{\ketDel*{#1}}
\newcommand{\ketD}{\ket{\fdot}}
\DeclarePairedDelimiterX{\braketDel}[2]{\langle}{\rangle}{#1 \:\delimsize\vert\: #2}
\newcommand{\braket}[2]{\braketDel*{#1}{#2}}


% Equivalence class
\DeclarePairedDelimiterXPP{\eqclDel}[2]{}{\lbrack}{\rbrack}{_{#2}}{#1}
\newcommand{\eqcl}[2][]{\eqclDel*{#2}{#1}}
\newcommand{\eqclD}[1][]{\eqclDel*{\fdot}{#1}}

% Set
\providecommand{\given}{} % Definiert ein leeren Command \given, damit er im \set Befehl stets neu definiert werden kann
\DeclarePairedDelimiterX{\setDel}[1]{\lbrace}{\rbrace}{\renewcommand{\given}{\:\delimsize\vert\:}#1}
\def\set{\@ifstar\@set\@@set}
\newcommand{\@@set}[1]{\setDel*{#1}}
\newcommand{\@set}[1]{\setDel{#1}}

% Folge
% M = nicht automatisch _#2 im inneren der Klammer
% MM = auch nicht das n\in\numN am Ende automatisch
\DeclarePairedDelimiterXPP{\seqDel}[2]{}{(}{)}{%
    \def\tmp{#2}
    \ifx\tmp\@nnil
         \def\tmp{}
     \else
         _{#2}
     \fi
}{
\def\tmp{#2}
    \ifx\tmp\@nnil
         #1,\ldots
     \else
         #1
     \fi
}

\def\seq{\@ifstar\@seq\@@seq}
\def\seqM{\@ifstar\@seqM\@@seqM}
\def\seqMM{\@ifstar\@seqMM\@@seqMM}
\def\subseq{\@ifstar\@subseq\@@subseq}
\newcommand{\@@seq}[2][n]{\seqDel*{#2_{#1}}{#1}}
\newcommand{\@seq}[2][n]{\seqDel{#2_{#1}}{#1}}
\newcommand{\@@seqM}[2][n]{\seqDel*{#2}{#1}}
\newcommand{\@seqM}[2][n]{\seqDel{#2}{#1}}
\newcommand{\@@seqMM}[2][\@nil]{\seqDel*{#2}{#1}}
\newcommand{\@seqMM}[2][\@nil]{\seqDel{#2}{#1}}
\newcommand{\@@subseq}[3][k]{\seqDel*{#2_{#3_{#1}}}{#1}}%TODO two optional arguments
\newcommand{\@subseq}[3][k]{\seqDel{#2_{#3_{#1}}}{#1}}%TODO two optional arguments
\makeatother

%%% Maps
\newcommand{\fun}[5][x]{
 \begin{aligned}
  &#2 \colon &\!\!\!\! #3 &\to #4\\
  &&\!\!\!\!#1 &\mapsto #5
 \end{aligned}
}


