% Default math stuff. See https://www.ctan.org/pkg/mathtools
\RequirePackage{mathtools}

% Add additional math symbols.
\RequirePackage{amssymb}%TODO For what exactly is it needed

% Add font to define the indicator function. See https://www.ctan.org/tex-archive/fonts/doublestroke
\RequirePackage{dsfont}

% Make the look of the text prettier. See https://www.namsu.de/Extra/pakete/Lmodern.html
\RequirePackage{lmodern}

% To make simple graphics. See https://tikz.dev/
\RequirePackage{tikz}

% Center a colon to the mathematical axis. An ordinary colon can be output with \ordinarycolon
\mathtoolsset{centercolon}
\newcommand{\coloneqq}{:=}
\renewcommand{\colon}{:}

% \DeclareTMT{<name>}{<english>}{<german>} Create a macro \text<name> which is translatable. Additionally create a macro \t<name> which is the translateable text in a text enviroment and a space infront. To make a space after the text use \t<name>\ or \t<name>{}
\newcommand{\DeclareTMT}[3]{%
    \edef\tmp{{ngerman}{\csname text#1\endcsname}{#3}}
    \expandafter\def\csname text#1\endcsname{#2}%
    \expandafter\defcaptionname\tmp%
    \expandafter\def\csname t#1\endcsname{\text{ \csname text#1\endcsname}}%
}

\DeclareTMT{on}{on}{auf}
\DeclareTMT{of}{of}{von}
\DeclareTMT{and}{and}{und}
\DeclareTMT{or}{or}{oder}
\DeclareTMT{in}{in}{in}
\DeclareTMT{for}{for}{für}
\DeclareTMT{forall}{for all}{für alle}
\DeclareTMT{with}{with}{mit}
\DeclareTMT{if}{if}{falls}
\DeclareTMT{else}{else}{sonst}
\DeclareTMT{even}{even}{gerade}
\DeclareTMT{odd}{odd}{ungerade}



%%%%%%%%%%%%%%%%%%%%%%%%%%%%%%%%%%%%
%%% Symbols
%%%%%%%%%%%%%%%%%%%%%%%%%%%%%%%%%%%%


%%% Greek letters
\renewcommand{\epsilon}{\varepsilon}
\renewcommand{\phi}{\varphi}
\renewcommand{\theta}{\vartheta}
\renewcommand{\rho}{\varrho}


%%% Upper d for integrals
\renewcommand{\d}{\mathop{}\!\mathrm{d}}


%%% Infinity
\newcommand{\oo}{\infty}

%%% Logical operators
\newcommand{\lif}{\Longrightarrow}
\newcommand{\lfi}{\Longleftarrow}
\newcommand{\liff}{\Longleftrightarrow}

%%% sigma-
\newcommand{\sigmad}{\ensuremath{\sigma}-}


%%% ^star
\renewcommand{\star}{\ensuremath{^\ast}}


%%% Dot to use in functions.
\newcommand{\fdot}{\:\cdot\:}


%%% Number spaces
\newcommand{\field}[1]{\mathbb{#1}}
\newcommand{\numN}{\field N}
\newcommand{\N}{\numN}
\newcommand{\numNo}{\numN_0}

\newcommand{\numZ}{\field Z}
\newcommand{\Z}{\numZ}

\newcommand{\numQ}{\field Q}
\newcommand{\Q}{\numQ}

\newcommand{\numR}{\field R}
\newcommand{\R}{\numR}
\newcommand{\numRp}{\numR^+}
\newcommand{\numRpo}{\numRp_0}

\newcommand{\numC}{\field C}
\newcommand{\C}{\numC}

\newcommand{\numK}{\field K}
\newcommand{\K}{\numK}


%%% Sets
\renewcommand{\complement}{\mathsf{c}}
\newcommand{\comp}{\complement}
\newcommand{\closure}[1]{\overline{#1}}
\newcommand{\inner}[1]{#1^\circ}
\newcommand{\restrict}[1]{\vert_{#1}}
\newcommand{\subsubset}{\subset\subset}
\newcommand{\relcomp}{\subsubset}
\newcommand{\embed}{\hookrightarrow}
\newcommand{\cembed}{\embed\embed}
\newcommand{\cupd}{\mathbin{\mathaccent\cdot\cup}}
\newcommand{\cupdot}{\cupd}


%%% Symbols with limits
% Convergence
\newcommand{\conv}[1][]{\xrightarrow{\:#1\:}}
\newcommand{\convn}[1][n]{\conv[#1\to\oo]}
\newcommand{\wconv}[1][]{\xrightharpoonup{\:#1\:}}
\newcommand{\wconvn}[1][n]{\wconv[#1\to\oo]}

% Lim
\newcommand{\limn}[1][n]{\lim_{#1\to\infty}}
\newcommand{\limnl}[1][n]{\lim\limits_{#1\to\infty}}

% Liminf
\newcommand{\liminfn}[1][n]{\liminf_{#1\to\infty}}
\newcommand{\liminfnl}[1][n]{\liminf\limits_{#1\to\infty}}

% Limsup
\newcommand{\limsupn}[1][n]{\limsup_{#1\to\infty}}
\newcommand{\limsupnl}[1][n]{\limsup\limits_{#1\to\infty}}

% Inf
\newcommand{\infn}[1][n]{\inf_{#1\in\numN}}
\newcommand{\infnl}[1][n]{\inf\limits_{#1\in\numN}}

% Sup
\newcommand{\supn}[1][n]{\sup_{#1\in\numN}}
\newcommand{\supnl}[1][n]{\sup\limits_{#1\in\numN}}

% Sum
\newcommand{\sumn}[1][n]{\sum_{#1=1}^\oo}
\newcommand{\sumnl}[1][n]{\sum\limits_{#1=1}^\oo}


%%% Function spaces
% Continuous functions
\newcommand{\funC}{\mathcal C}
\newcommand{\funCk}[1][k]{\funC^{#1}}
\newcommand{\funCko}[1][k]{\funCk[#1]_0}
\newcommand{\funCo}{\funCk[0]}
\newcommand{\funCoo}{\funCk[\oo]}
\newcommand{\funCooo}{\funCko[\oo]}
\newcommand{\funClip}{\funCk[\text{Lip}]}
\newcommand{\funCb}{\funC_\text{b}}
\newcommand{\funCkg}[1][k,\gamma]{\funCk[#1]}
\newcommand{\funCog}[1][0,\gamma]{\funCk[#1]}

% Polynomials
\newcommand{\funP}{\mathcal P}
\newcommand{\funPn}[1][n]{\funP_{#1}}

% Lp
\newcommand{\funL}{L}
\newcommand{\funLp}[1][p]{\funL^{#1}}
\newcommand{\funLoo}{\funLp[\oo]}
\newcommand{\funLploc}[1][p]{\funL^{#1}_\text{loc}}

\newcommand{\funl}{\mathcal L}
\newcommand{\funlp}[1][p]{\funl^{#1}}
\newcommand{\funloo}{\funlp[\oo]}
\newcommand{\funlploc}[1][p]{\funl^{#1}_\text{loc}}

% Wkp
\newcommand{\funW}{W}
\newcommand{\funWkp}[1][k,p]{\funW^{#1}}
\newcommand{\funWkpo}[1][k,p]{\funWkp[#1]_0}
\newcommand{\funWkploc}[1][k,p]{\funWkp[#1]_\text{loc}}

% Hilbert spaces
\newcommand{\funH}{H}
\newcommand{\funHk}[1][k]{\funH^{#1}}
\newcommand{\funHko}[1][k]{\funH[#1]_0}

% Distributions
\newcommand{\funD}{\mathcal D}
\newcommand{\funS}{\mathcal S}

% Factor space
\newcommand{\factor}[2]{{}^{#1}\!/\!_{#2}}

% Sequence spaces
\newcommand{\seqlp}[1][p]{\ell^{#1}}
\newcommand{\seqwkp}[1][k,p]{w^{#1}}
\newcommand{\seqc}{c}
\newcommand{\seqco}{\seqc_0}
\newcommand{\seqd}{d}


%%%Integral

% \Xint{symbol} integral with symbol in it
\def\Xint#1{\mathchoice
{\XXint\displaystyle\textstyle{#1}}%
{\XXint\textstyle\scriptstyle{#1}}%
{\XXint\scriptstyle\scriptscriptstyle{#1}}%
{\XXint\scriptscriptstyle\scriptscriptstyle{#1}}%
\!\int}
\def\XXint#1#2#3{{\setbox0=\hbox{$#1{#2#3}{\int}$} \vcenter{\hbox{$#2#3$}}\kern-.5\wd0}}

% Averrage integral
\def\aint{\Xint{\text{\hspace{.005\wd0}\tiny\rotatebox[origin=c]{-45}{/}}}}


%%%%%%%%%%%%%%%%%%%%%%%%%%%%%%%%%%%%
%%% Math operators
%%%%%%%%%%%%%%%%%%%%%%%%%%%%%%%%%%%%


% Real part of a complex number
\let\Re\relax
\DeclareMathOperator{\Re}{Re}

% Imaginary part of a complex number
\let\Im\relax
\DeclareMathOperator{\Im}{Im}

% Signum function
\DeclareMathOperator{\sgn}{sgn}

% Linear Hull
\DeclareMathOperator{\lin}{span}

% Arcus cotangens
\DeclareMathOperator{\arccot}{arccot}

% Graph
\DeclareMathOperator{\graph}{graph}

% Range
\DeclareTMT{ran}{ran}{Bild}% Defines \textran
\DeclareMathOperator{\ran}{\textran}

% Support
\DeclareMathOperator{\supp}{supp}

% Trace
\DeclareTMT{tr}{tr}{spur}
\DeclareMathOperator{\tr}{\texttr}

% Identity
\DeclareMathOperator{\id}{id}

% Rank
\DeclareTMT{rk}{rk}{rg}
\DeclareMathOperator{\rk}{\textrk}

% Distance
\DeclareMathOperator{\dist}{dist}

% Ddiameter
\DeclareMathOperator{\diam}{diam}

% Indicator function
\DeclareMathOperator{\ind}{\mathds{1}}

% Cofactor matrix
\DeclareMathOperator{\cof}{cof}

% Accumulation values
\DeclareTMT{av}{av}{HW}
\DeclareMathOperator{\av}{\textav}

% Accumulation points
\DeclareTMT{ap}{ap}{HP}
\DeclareMathOperator{\ap}{\textap}

% Gradient
\DeclareMathOperator{\grad}{\nabla}
\DeclareMathOperator{\tgrad}{grad}

% Divergence
\let\div\relax
\DeclareMathOperator{\div}{\grad \cdot}
\DeclareMathOperator{\tdiv}{div}

% Curl product
\newcommand{\curl}{\times}

% Rotation
\DeclareMathOperator{\rot}{\grad \curl}
\DeclareMathOperator{\trot}{rot}

% Laplace
\DeclareMathOperator{\laplace}{\Delta}


%%%%%%%%%%%%%%%%%%%%%%%%%%%%%%%%%%%%
%%% Paired delimiters
%%%%%%%%%%%%%%%%%%%%%%%%%%%%%%%%%%%%
\makeatletter


% Absolut value
\DeclarePairedDelimiterXPP{\absDel}[2]{}{\lvert}{\rvert}{_{#2}}{#1}
\def\abs{\@ifstar\@abs\@@abs}
\def\absD{\@ifstar\@absD\@@absD}
\newcommand{\@@abs}[2][]{\absDel*{#2}{#1}}
\newcommand{\@@absD}[1][]{\absDel*{\fdot}{#1}}
\newcommand{\@abs}[2][]{\absDel{#2}{#1}}
\newcommand{\@absD}[1][]{\absDel{\fdot}{#1}}

% Norm with ||
\DeclarePairedDelimiterXPP{\normDel}[2]{}{\lVert}{\rVert}{_{#2}}{#1}
\def\norm{\@ifstar\@norm\@@norm}
\def\normD{\@ifstar\@normD\@@normD}
\def\normop{\@ifstar\@normop\@@normop}
\def\normopD{\@ifstar\@normopD\@@normopD}
\newcommand{\@@norm}[2][]{\normDel*{#2}{#1}}
\newcommand{\@@normD}[1][]{\normDel*{\fdot}{#1}}
\newcommand{\@@normop}[1][]{\normDel*{#1}{\text{op}}}
\newcommand{\@@normopD}{\normDel*{\fdot}{\text{op}}}
\newcommand{\@norm}[2][]{\normDel{#2}{#1}}
\newcommand{\@normD}[1][]{\normDel{\fdot}{#1}}
\newcommand{\@normop}[1]{\normDel{#1}{\text{op}}}
\newcommand{\@normopD}{\normDel{\fdot}{\text{op}}}

% Norm with |||
\DeclareFontFamily{U}{matha}{\hyphenchar\font45}
\DeclareFontShape{U}{matha}{m}{n}{
 <5> <6> <7> <8> <9> <10> gen * matha <10.95> matha10 <12> <14.4> <17.28> <20.74> <24.88> matha12
}{}
\DeclareSymbolFont{matha}{U}{matha}{m}{n}
\DeclareFontSubstitution{U}{matha}{m}{n}
\DeclareFontFamily{U}{mathx}{\hyphenchar\font45}
\DeclareFontShape{U}{mathx}{m}{n}{
 <5> <6> <7> <8> <9> <10> <10.95> <12> <14.4> <17.28> <20.74> <24.88> mathx10
}{}
\DeclareSymbolFont{mathx}{U}{mathx}{m}{n}
\DeclareFontSubstitution{U}{mathx}{m}{n}
\DeclareMathDelimiter{\vvvert}{0}{matha}{"7E}{mathx}{"17}
\DeclarePairedDelimiterXPP{\NormDel}[2]{}{\vvvert}{\vvvert}{_{#2}}{#1}
\def\Norm{\@ifstar\@Norm\@@Norm}
\def\NormD{\@ifstar\@NormD\@@NormD}
\newcommand{\@@Norm}[2][]{\NormDel*{#2}{#1}}
\newcommand{\@@NormD}[1][]{\NormDel*{\fdot}{#1}}
\newcommand{\@Norm}[2][]{\NormDel{#2}{#1}}
\newcommand{\@NormD}[1][]{\NormDel{\fdot}{#1}}

% Scalar product
\DeclarePairedDelimiterXPP{\innerprodDel}[3]{}{\langle}{\rangle}{_{#3}}{#1,#2}
\def\innerprod{\@ifstar\@innerprod\@@innerprod}
\def\innerprodD{\@ifstar\@innerprodD\@@innerprodD}
\newcommand{\@@innerprod}[3][]{\innerprodDel*{#2}{#3}{#1}}
\newcommand{\@@innerprodD}[1][]{\innerprodDel*{\fdot}{\fdot}{#1}}
\newcommand{\@innerprod}[3][]{\innerprodDel{#2}{#3}{#1}}
\newcommand{\@innerprodD}[1][]{\innerprodDel{\fdot}{\fdot}{#1}}


% Commutator
\DeclarePairedDelimiterX{\commuDel}[2]{\lbrack}{\rbrack}{#1,#2}
\newcommand{\commutator}[2]{\commuDel*{#1}{#2}}
\newcommand{\commutatorbf}[2]{\commuDel*{\mathbf #1}{\mathbf #2}}


% Bra-ket
\DeclarePairedDelimiter{\braDel}{\langle}{\vert}
\newcommand{\bra}[1]{\braDel*{#1}}
\newcommand{\braD}{\bra{\fdot}}
\DeclarePairedDelimiter{\ketDel}{\vert}{\rangle}
\newcommand{\ket}[1]{\ketDel*{#1}}
\newcommand{\ketD}{\ket{\fdot}}
\DeclarePairedDelimiterX{\braketDel}[2]{\langle}{\rangle}{#1 \:\delimsize\vert\: #2}
\newcommand{\braket}[2]{\braketDel*{#1}{#2}}


% Equivalence class
\DeclarePairedDelimiterXPP{\eqclDel}[2]{}{\lbrack}{\rbrack}{_{#2}}{#1}
\newcommand{\eqcl}[2][]{\eqclDel*{#2}{#1}}
\newcommand{\eqclD}[1][]{\eqclDel*{\fdot}{#1}}

% Set
\providecommand{\given}{} % Definiert ein leeren Command \given, damit er im \set Befehl stets neu definiert werden kann
\DeclarePairedDelimiterX{\setDel}[1]{\lbrace}{\rbrace}{\renewcommand{\given}{\:\delimsize\vert\:}#1}
\def\set{\@ifstar\@set\@@set}
\newcommand{\@@set}[1]{\setDel*{#1}}
\newcommand{\@set}[1]{\setDel{#1}}

% Folge
% M = nicht automatisch _#2 im inneren der Klammer
% MM = auch nicht das n\in\numN am Ende automatisch
\DeclarePairedDelimiterXPP{\seqDel}[2]{}{(}{)}{%
    \def\tmp{#2}
    \ifx\tmp\@nnil
         \def\tmp{}
     \else
         _{#2}
     \fi
}{
\def\tmp{#2}
    \ifx\tmp\@nnil
         #1,\ldots
     \else
         #1
     \fi
}

\def\seq{\@ifstar\@seq\@@seq}
\def\seqM{\@ifstar\@seqM\@@seqM}
\def\seqMM{\@ifstar\@seqMM\@@seqMM}
\def\subseq{\@ifstar\@subseq\@@subseq}
\newcommand{\@@seq}[2][n]{\seqDel*{#2_{#1}}{#1}}
\newcommand{\@seq}[2][n]{\seqDel{#2_{#1}}{#1}}
\newcommand{\@@seqM}[2][n]{\seqDel*{#2}{#1}}
\newcommand{\@seqM}[2][n]{\seqDel{#2}{#1}}
\newcommand{\@@seqMM}[2][\@nil]{\seqDel*{#2}{#1}}
\newcommand{\@seqMM}[2][\@nil]{\seqDel{#2}{#1}}
\newcommand{\@@subseq}[3][k]{\seqDel*{#2_{#3_{#1}}}{#1}}%TODO two optional arguments
\newcommand{\@subseq}[3][k]{\seqDel{#2_{#3_{#1}}}{#1}}%TODO two optional arguments
\makeatother

%%% Maps
\newcommand{\fun}[5][x]{
 \begin{aligned}
  &#2 \colon &\!\!\!\! #3 &\to #4\\
  &&\!\!\!\!#1 &\mapsto #5
 \end{aligned}
}

